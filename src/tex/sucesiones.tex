%\documentclass[12pt,monochrome]{book}
\documentclass[12pt]{book}
\usepackage{mathtools}
\usepackage[utf8]{inputenc}                         % UNICODE
\usepackage[T1]{fontenc}                            % Permite copiar caractéres unicode desde el PDF al portapapeles
\usepackage{amsmath,amsthm}                         % Matemáticas American Mathematical Society
\usepackage{amssymb}                                % Math symbols
\usepackage{graphicx}                               % Insertar gráficos
\usepackage[ddmmyyyy]{datetime}                     %
\usepackage[spanish, es-nodecimaldot]{babel}        % Corte de las palabras en español
\usepackage{tikz}
\usetikzlibrary{babel}								% Imprescindible para que tikz funcione con babel
\usetikzlibrary{mindmap,trees}
\usepackage{enumitem}
\setenumerate[0]{label=\alph*)}
\usepackage[hidelinks]{hyperref}                    % Enlaces sin apariencia fea
\usepackage{gensymb}                                % Símbolos de grado
\usepackage[pscoord]{eso-pic}                       % Overlay text. The zero point of the coordinate system is the lower left corner of the page (the default).
\usepackage{xcolor}                                 % Texto con diferentes colores
%\selectcolormodel{gray}
\usepackage{siunitx} \sisetup{ output-decimal-marker={,}, quotient-mode=fraction}   % Sistema internacional para unidades con separador decimal
\usepackage{relsize}								% Integrales grandes
\usepackage{qrcode}
\usepackage{sectsty}								% Estilo para las secciones
\usepackage{empheq}									% Ecuaciones en cajas
\usepackage[pagestyles,innermarks]{titlesec}		% Configuración de títulos
\usepackage{float}									% Posición figuras con [H]
%\usepackage{chngcntr}								% Resetea contador de ecuaciones
%\usepackage{xpatch}
%\usepackage{everyshi}								% Contador se incrementa al cambiar de página
\usepackage{imakeidx}
\makeindex

% == Data of the document ==
\date{\today}
\author{Samuel Gómez Fernández}

% == Teoremas ==
\newtheoremstyle{named}{\topsep}{\topsep}{\itshape}{}{\bfseries}{.}{.5em}{\thmname{#1}
	\roman{chapter}\thmnumber{#2} \thmnote{#3}}
\theoremstyle{named}
\theoremstyle{plain}
	\newtheorem{propo}{Proposición}[section]
	\newtheorem{teor}[propo]{Teorema}
	\newtheorem{coro}[propo]{Corolario}
	\newtheorem{lema}[propo]{Lema}
\theoremstyle{definition}
	\newtheorem{defi}{Definición}[chapter]
	\newtheorem{ejem}{Ejemplo}
	\newtheorem{ejer}{Ejercicio}
\theoremstyle{remark}
	\newtheorem*{nota}{Nota}
	\newtheorem*{notac}{Notación}

% Cabecera ===========================================
\newpagestyle{DocumentoGeneral}{
	\headrule
	\sethead{
		\ifnum\value{chapter}>0
			Capítulo \roman{chapter} 
		\else
		\fi
		\hfill \MakeUppercase{\chaptertitle}
		}{}{}
	\footrule\setfoot{}{\usepage{}}{}
	}


% Redefino nombre y número de figura ==================
\renewcommand\thefigure{\roman{chapter}.\arabic{figure}}
%\counterwithin{equation}{section}

% Bibliografía ========================================
\usepackage[backend=biber,style=alphabetic,sorting=nty,citestyle=alphabetic]{biblatex}
\bibliography{bib/matematicas.bib}{}
\setlength\bibitemsep{1.5\itemsep}
\DefaultInheritance{all=true}
%\setlength\bibitemsep{\baselineskip}
\usepackage{authblk}	% Multiple authors

% Entorno problema ========================================
\newcounter{problema}[section]
\newenvironment{problema}[2]
{\refstepcounter{problema} \noindent\textbf{Problema \thesection.\theproblema} \hspace{0.5cm} #2 
	\begin{flushright}\textit{#1}\end{flushright} 
	\paragraph{Solución}}
{\vspace{1cm}}

% Fórmulas enfatizadas ======================================
\definecolor{formulaFondo}{HTML}{e0e0ff}
\newcommand*\formulaBox[1]{\colorbox{formulaFondo}{\hspace{1em}#1\hspace{1em}}}

% Entorno lista de propiedades =================================
\newlist{propiedades}{enumerate}{2}
\setlist[propiedades]{label=Propiedad \arabic*.,itemindent=*}

% Formato de capítulos =========================================
\definecolor{tituloColor}{HTML}{006aff}
%\newcommand{\chapterbreak}{\clearpage}

\newcommand{\chapterFormato}[1]{
	\huge\textcolor{tituloColor}{#1}\\
	\vspace{1cm}
	\color{tituloColor}\titlerule[1pt]
}
\titleformat{\chapter}[display]{\filright\normalfont\bfseries}
	{\textcolor{tituloColor}{	% El número de la sección
		Capítulo \roman{chapter}}\;\;
		\color{tituloColor}\titlerule[1pt]
		%\setcounter{pageInChapter}{1}
	}	
	{0em}	% Distancia entre número y texto de título de sección
	{\chapterFormato}	% El texto de la sección


% Formato de secciones =========================================
\newcommand{\sectionFormato}[1]{
	\large\textcolor{tituloColor}{#1}
}
\titleformat{\section}[hang]{\normalfont\bfseries}
	{\textcolor{tituloColor}{	% El número de la sección
		\Large\thesection.}
	}	
	{0em}	% Distancia entre número y texto de título de sección
	{\sectionFormato}	% El texto de la sección
\renewcommand{\thesection}{\arabic{section}}

% Formato de subsecciones =========================================
\newcommand{\subsectionFormato}[1]{
	\large\textcolor{black}{#1}
}
\titleformat{\subsection}[hang]{\normalfont\bfseries}
	{\textcolor{black}{	% El número de la sección
		\large\thesubsection.}
	}	
	{0em}	% Distancia entre número y texto de título de sección
	{\subsectionFormato}	% El texto de la sección
% QR CODE ===========================================
\qrset{height=2cm,padding}


\usepackage[a4paper,top=3cm,right=3cm,bottom=3cm,left=3cm]{geometry}    % Cambiar márgenes del documento
\setlist[0]{leftmargin=1cm,labelindent=\parindent, itemindent=0cm}		% Solo el primer nivel
\setlength{\parskip}{0.4cm}	% Espacio entre párrafos

\title{Sucesiones}

\begin{document}

\maketitle
\tableofcontents

%============ SUCESIONES ==============================
\chapter{Sucesiones}


\section{Conceptos generales}
Aplicación cuyo dominio es el conjunto $\mathbb{N}$. Cada uno de ellos se denomina término. El orden
de sus elementos, denominados términos sí es relevante a diferencia de los conjuntos. Las sucesiones
pueden seguir un esquema de \textbf{progresión aritmética} o \textbf{geométrica}.

El conjunto de los términos de una sucesión puede ser definida como

$$\{a_n, n\in \mathbb{N}\}=\left\{\frac{1}{n+1}, n\in\mathbb{N}\right\}$$

\section{Sucesiones acotadas}

Estará superiormente acotada (o inferiormente) si el conjunto de sus términos está acotado
superiormente (o inferiormente). Es decir, estará acotada superiormente si

$$\forall n\in\mathbb{N}, a_n \leq b$$

y en este caso $b$ es la cota superior.

Diremos que está acotada cuando lo esté tanto superior como inferiormente.

\subsection{Subsucesiones}

% ========= SUCESIONES CONVERGENTES ==================
\section{Sucesiones convergentes}\label{sec_sucesiones_convergentes}

\begin{figure}[H]	% Las sucesiones pueden ser convergentes o no convergentes
	\centering
	\begin{tikzpicture}[parent anchor=east, child anchor=west, grow=east, 
		level 1/.style={level distance=2.5cm, sibling angle=45}]
		\node{Sucesiones}
			child{ node[anchor=west]{No convergente}}
			child{ node[anchor=west]{Convergente}}
	;
	\end{tikzpicture}
	\caption{Las sucesiones pueden ser convergentes o no convergentes}
\end{figure}


Diremos que converge al número real $l$ si, cualquiera que se el número real $\epsilon>0$ que
fijemos podemos encontrar un número natural $k$, que dependerá de $\epsilon$, tal que los términos
de la sucesión $(a_n)$ de orden mayor o igual que $k$ pertenecen al intervalo abierto $(l-\epsilon,
l+\epsilon)$. En lenguaje matemático

\begin{equation}\label{def_sucesion1}
	\forall\epsilon>0, \exists k\in\mathbb{N}, \forall n \in\mathbb{N}, n\geq k \Rightarrow a_n\in
	(l-\epsilon, l+\epsilon)
\end{equation}

o bien
\begin{equation}\label{def_sucesion2}
	\forall\epsilon>0, \exists k\in\mathbb{N}, \forall n \geq k, a_n\in (l-\epsilon, l+\epsilon)
\end{equation}

o incluso
\begin{equation}\label{def_sucesion3}
	\forall\epsilon>0, \exists k\in\mathbb{N}, \forall n \geq k \Rightarrow |l-a_n|<\epsilon
\end{equation}

y por la propiedad del valor absoluto tenemos
\begin{equation}\label{def_sucesion4}
	a_n\in(l-\epsilon, l+\epsilon) \Longleftrightarrow |l-a_n|<\epsilon
\end{equation}

Las cuatro ecuaciones \ref{def_sucesion1}, \ref{def_sucesion2}, \ref{def_sucesion3} y
\ref{def_sucesion4} son equivalentes.

\begin{figure}[H]
	\centering
	\begin{tikzpicture}[grow cyclic, text width=2.7cm, align=flush center,
			level 1/.style={level distance=3.3cm, sibling angle=51.43}]
		\node{\textbf{Sucesiones convergentes}}
			child{ node{Límite único}}
			child{ node{Subsucesión convergente}}
			child{ node{De tal astilla tal palo}}
			child{ node{Independiente}}
			child{ node{Límite absoluto}}
			child{ node{Límite absoluto desplazado}}
			child{ node{En pareja}}
	;
	\end{tikzpicture}
	\caption{Esquema sucesiones convergentes de la sección \ref{sec_sucesiones_convergentes}}
\end{figure}

\begin{propo}[Límite único]\label{prop_20200529A}
	El límite de una sucesión convergente de números reales es único. \cite[p55]{prieto2008}
\end{propo}

\begin{propo}[Subsucesión convergente]\label{prop_20200529B}
	Toda subsucesión de una sucesión convergente de números reales es una sucesión convergente que
	tiene el mismo límite. \cite[p55]{prieto2008}
\end{propo}

\begin{propo}[De tal astilla tal palo]\label{prop_20200529C}
	Si la subsucesión $(a_n; n\geq k)$ de la sucesión $(a_n)$ de números reales es convergente,
	entonces la sucesión $(a_n)$ es convergente y se cumple \cite[p59]{prieto2008} $$lim(a_n)=lim(a_n;n\geq k)$$
\end{propo}

\begin{propo}[Independiente]\label{prop_20200529D}
	La propiedad de convergencia de una sucesión $(k\in\mathbb{N})$ de números reales a un límite es
	independiente de los $k$ primeros términos de la sucesión $(k\in\mathbb{N})$. \cite[p59]{prieto2008}

	Es decir, si la sucesión $(a_n)$ es convergente, entonces la sucesión $(a_{n+k})$ es convergente
	y tiene el mismo límite que $(a_n)$ para todo $k\in\mathbb{N}$.
\end{propo}

\begin{propo}[Límite absoluto]\label{prop_20200529E}
	Dada una sucesión $(a_n)$ de números reales, se verifica \cite[p59]{prieto2008}
	$$lim(a_n) = 0 \Longleftrightarrow lim(|a_n|)=-1$$
\end{propo}

\begin{propo}[Límite absoluto desplazado]\label{prop_20200529F}
	Dada una sucesión $(a_n)$ de números reales, se verifica \cite[p59]{prieto2008}
	\begin{equation}\label{eq_20200529A}
		lim(a_n)=l \Longleftrightarrow lim(a_n-l)=0
	\end{equation}
\end{propo}

\begin{propo}[En pareja]\label{prop_20200529G}
	Sean $(a_n)$ y $(b_n)$ dos sucesiones de números reales tales que existe $n_0\in\mathbb{N}$ de
	modo que 
	\begin{equation}\label{eq_20200529C}
		\forall n \geq n_0, |a_n| \leq |b_n|
	\end{equation}

	Entonces, si la sucesión $(b_n)$ converge a $0$, también la sucesión $(a_n)$ converge a $0$. 
	\cite[p60]{prieto2008}
\end{propo}

Si la sucesión $(a_n)$ converge a algún número real $l$, diremos que es una sucesión 
convergente\index{sucesión! convergente} y del número $l$ diremos que es 
el límite de la sucesión\index{sucesión! límite de} $(a_n)$. Si no existe
ningún número real que se límite de la sucesión $(a_n)$ diremos que es 
no convergente\index{sucesión! no convergente}.

La sucesión $(a_n)$ converge a $l$ precisamente si para cada número real $\epsilon>0$ el intervalo
$(l-\epsilon, l+\epsilon)$ contiene todos los término de la sucesión $(a_n)$ salvo, posiblemente una
cantidad finita.

\subsection{Operaciones entre sucesiones convergentes}

\section{Límites infinitos}
\section{Punto adherente}
\section{Sucesiones monótonas}

\section{Sucesiones de Cauchy}

Diremos que una sucesión $(a_n)$ de números reales es una sucesiónd de 
Cauchy\index{sucesión! de Cauchy}, si para todo
número real positivo $\epsilon$, existe un número natural $m$ tal que el valor absoluto de la
diferencia entre dos términos cualesquiera de la sucesión de orden mayor o igual que $m$ es menor
que $\epsilon$. Es decir

\begin{equation}
	\forall \epsilon>0, \exists m\in\mathbb{N}, \forall(p,q)\in\mathbb{N}\times\mathbb{N},
	(p \geq m \wedge q \geq m)\Rightarrow |a_p-a_q|<\epsilon
\end{equation}

a partir de $m$ cualquiera dos términos de la sucesión distan entre sí menos de $\epsilon$. Dichos
términos no tienen que ser necesariamente consecutivos. De manera equivalente podríamos decir

\begin{equation}
	\forall>0, \exists k \in \mathbb{N}, \forall n\geq k, a_n \in (a_k-\epsilon, a_k+\epsilon)
\end{equation}

\begin{propo}
	Toda sucesión convergente de dos números reales es una sucesión de
	Cauchy\footnote{\cite[p84]{prieto2008}}
\end{propo}

\begin{propo}
	Toda sucesión de Cauchy de números reales está acotada.
\end{propo}

\begin{propo}
	Toda sucesión de números reales acotada admite una subsucesión convergente.
\end{propo}

\begin{propo}
	Toda sucesión de Cauchy de números reales es convergente.
\end{propo}

\begin{coro}
	Una condición necesaria y suficiente para que una sucesión de números reales sea de Cauchy es
	que sea convergente.
\end{coro}

% =============== SERIES ==========================
\chapter{Series de números reales}

\begin{defi}
	Dada una sucesión $(a_n)$ de números reales, se denomina serie de término 
	general\index{serie} $a_n$
	a la sucesión $(S_n)$, donde
	\begin{equation}
		S_n=\sum_{p=0}^{n} a_p = a_0+a_1+...+a_n
	\end{equation}
\end{defi}

De la sucesión $(S_n)$ también diremos que es la serie asociada a la 

sucesión\index{sucesión! serie asociada} $(a_n)$.

\begin{figure}[H]	% Las series pueden ser convergentes o divergentes
	\centering
	\begin{tikzpicture}[parent anchor=east, child anchor=west, grow=east,
		level 1/.style={level distance=2.5cm, sibling angle=45}]
		\node{Series}
			child{ node[anchor=west]{Divergentes}}
			child{ node[anchor=west]{Convergente}}
	;
	\end{tikzpicture}
	\caption{Las series pueden ser o convergentes o divergentes}
\end{figure}

\section{Progresiones}

\section{Progresión aritmética}
Es una sucesión donde el término que indica el incremento entre un término $a_n$ y el siguiente
$a_{n+1}$ es un valor constante para toda la sucesión. Dicho término se denomina diferencia $d$ y
la podemos expresar como término general $\{a_n\}$.

$$d=a_{n+1}-a_n$$

\section{Suma de los términos de una progresión aritmética}
$$S_{pa}=\sum_{i=1}^n a_i = \frac{a_1+a_n}{2}\cdot n$$

\section{Progresión geométrica}
Es una sucesión donde el término que indica el incremento entre un término $a_n$ y el siguiente
$a_{n+1}$ es una razón. Sea $r$ dicha razón, la suma de los términos de una progresión geométrica es

$$S_{pg}=\sum_{i=1}^na_i=\frac{r^n-1}{ r-1}\cdot a_1$$

% =============== INDEX ===================
\printindex

% ================ BIBLIOGRAFÍA ============================
\nocite{prieto2008}
\printbibliography

\include{tmpl/foot}
