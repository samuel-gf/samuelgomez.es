%\documentclass[12pt,monochrome]{book}
\documentclass[12pt]{book}
\usepackage{mathtools}
\usepackage[utf8]{inputenc}                         % UNICODE
\usepackage[T1]{fontenc}                            % Permite copiar caractéres unicode desde el PDF al portapapeles
\usepackage{amsmath,amsthm}                         % Matemáticas American Mathematical Society
\usepackage{amssymb}                                % Math symbols
\usepackage{graphicx}                               % Insertar gráficos
\usepackage[ddmmyyyy]{datetime}                     %
\usepackage[spanish, es-nodecimaldot]{babel}        % Corte de las palabras en español
\usepackage{tikz}
\usetikzlibrary{babel}								% Imprescindible para que tikz funcione con babel
\usetikzlibrary{mindmap,trees}
\usepackage{enumitem}
\setenumerate[0]{label=\alph*)}
\usepackage[hidelinks]{hyperref}                    % Enlaces sin apariencia fea
\usepackage{gensymb}                                % Símbolos de grado
\usepackage[pscoord]{eso-pic}                       % Overlay text. The zero point of the coordinate system is the lower left corner of the page (the default).
\usepackage{xcolor}                                 % Texto con diferentes colores
%\selectcolormodel{gray}
\usepackage{siunitx} \sisetup{ output-decimal-marker={,}, quotient-mode=fraction}   % Sistema internacional para unidades con separador decimal
\usepackage{relsize}								% Integrales grandes
\usepackage{qrcode}
\usepackage{sectsty}								% Estilo para las secciones
\usepackage{empheq}									% Ecuaciones en cajas
\usepackage[pagestyles,innermarks]{titlesec}		% Configuración de títulos
\usepackage{float}									% Posición figuras con [H]
%\usepackage{chngcntr}								% Resetea contador de ecuaciones
%\usepackage{xpatch}
%\usepackage{everyshi}								% Contador se incrementa al cambiar de página
\usepackage{imakeidx}
\makeindex

% == Data of the document ==
\date{\today}
\author{Samuel Gómez Fernández}

% == Teoremas ==
\newtheoremstyle{named}{\topsep}{\topsep}{\itshape}{}{\bfseries}{.}{.5em}{\thmname{#1}
	\roman{chapter}\thmnumber{#2} \thmnote{#3}}
\theoremstyle{named}
\theoremstyle{plain}
	\newtheorem{propo}{Proposición}[section]
	\newtheorem{teor}[propo]{Teorema}
	\newtheorem{coro}[propo]{Corolario}
	\newtheorem{lema}[propo]{Lema}
\theoremstyle{definition}
	\newtheorem{defi}{Definición}[chapter]
	\newtheorem{ejem}{Ejemplo}
	\newtheorem{ejer}{Ejercicio}
\theoremstyle{remark}
	\newtheorem*{nota}{Nota}
	\newtheorem*{notac}{Notación}

% Cabecera ===========================================
\newpagestyle{DocumentoGeneral}{
	\headrule
	\sethead{
		\ifnum\value{chapter}>0
			Capítulo \roman{chapter} 
		\else
		\fi
		\hfill \MakeUppercase{\chaptertitle}
		}{}{}
	\footrule\setfoot{}{\usepage{}}{}
	}


% Redefino nombre y número de figura ==================
\renewcommand\thefigure{\roman{chapter}.\arabic{figure}}
%\counterwithin{equation}{section}

% Bibliografía ========================================
\usepackage[backend=biber,style=alphabetic,sorting=nty,citestyle=alphabetic]{biblatex}
\bibliography{bib/matematicas.bib}{}
\setlength\bibitemsep{1.5\itemsep}
\DefaultInheritance{all=true}
%\setlength\bibitemsep{\baselineskip}
\usepackage{authblk}	% Multiple authors

% Entorno problema ========================================
\newcounter{problema}[section]
\newenvironment{problema}[2]
{\refstepcounter{problema} \noindent\textbf{Problema \thesection.\theproblema} \hspace{0.5cm} #2 
	\begin{flushright}\textit{#1}\end{flushright} 
	\paragraph{Solución}}
{\vspace{1cm}}

% Fórmulas enfatizadas ======================================
\definecolor{formulaFondo}{HTML}{e0e0ff}
\newcommand*\formulaBox[1]{\colorbox{formulaFondo}{\hspace{1em}#1\hspace{1em}}}

% Entorno lista de propiedades =================================
\newlist{propiedades}{enumerate}{2}
\setlist[propiedades]{label=Propiedad \arabic*.,itemindent=*}

% Formato de capítulos =========================================
\definecolor{tituloColor}{HTML}{006aff}
%\newcommand{\chapterbreak}{\clearpage}

\newcommand{\chapterFormato}[1]{
	\huge\textcolor{tituloColor}{#1}\\
	\vspace{1cm}
	\color{tituloColor}\titlerule[1pt]
}
\titleformat{\chapter}[display]{\filright\normalfont\bfseries}
	{\textcolor{tituloColor}{	% El número de la sección
		Capítulo \roman{chapter}}\;\;
		\color{tituloColor}\titlerule[1pt]
		%\setcounter{pageInChapter}{1}
	}	
	{0em}	% Distancia entre número y texto de título de sección
	{\chapterFormato}	% El texto de la sección


% Formato de secciones =========================================
\newcommand{\sectionFormato}[1]{
	\large\textcolor{tituloColor}{#1}
}
\titleformat{\section}[hang]{\normalfont\bfseries}
	{\textcolor{tituloColor}{	% El número de la sección
		\Large\thesection.}
	}	
	{0em}	% Distancia entre número y texto de título de sección
	{\sectionFormato}	% El texto de la sección
\renewcommand{\thesection}{\arabic{section}}

% Formato de subsecciones =========================================
\newcommand{\subsectionFormato}[1]{
	\large\textcolor{black}{#1}
}
\titleformat{\subsection}[hang]{\normalfont\bfseries}
	{\textcolor{black}{	% El número de la sección
		\large\thesubsection.}
	}	
	{0em}	% Distancia entre número y texto de título de sección
	{\subsectionFormato}	% El texto de la sección
% QR CODE ===========================================
\qrset{height=2cm,padding}


\usepackage[a4paper,top=3cm,right=3cm,bottom=3cm,left=3cm]{geometry}    % Cambiar márgenes del documento
\setlist[0]{leftmargin=1cm,labelindent=\parindent, itemindent=0cm}		% Solo el primer nivel
\setlength{\parskip}{0.4cm}	% Espacio entre párrafos

\title{Recetas para tener éxito en geometría}

%index{término para el índice}
%\renewcommand\vec{\overline}
\setenumerate[1]{label=\arabic*.-}

\begin{document}

%\maketitle
\tableofcontents
\pagestyle{DocumentoGeneral}

%================================ CONCEPTOS GENERALES =====================================

\chapter{Conceptos generales}
\section{Punto}
Un elemento definido con coordenadas. Suelen recibir nombres como $P$ o $A$, 
y si queremos ser un poco más concretos, nombres más elaborados como $P_r$. 

En el plano tienen dos coordenadas $x$ e $y$ y en el espacio
tienen tres $x$, $y$ y $z$.

Para asociar las coordenadas al punto se suele escribir cosas como $\color{red}A(x,y,z)$ o
$\color{blue}P(x_2, y_2, z_2)$ y
así sabemos que $\color{red}x$ es la coordenada del punto $\color{red}A$ pero que $\color{blue}x_2$
es la coordenada del punto $\color{blue}P$.

Existe un punto especial llamado \textbf{origen} que tiene por coordenadas $(0,0)$ en el plano y
$(0,0,0)$ en el espacio.

%=== RECTA ===
\section{Rectas}

Es un conjunto infinito de puntos alineados. Hay varias formas de representarla analíticamente. Es evidente
que una recta $r$ puede ser recorrida en los dos sentidos opuestos.

\subsection{Ecuación vectorial en el espacio}
La forma vectorial de la recta presenta este aspecto
\begin{empheq}[box=\formulaBox]{equation}
	(x,y,z)=(x_0, y_0, z_0)+\lambda(a, b, c)
\end{empheq}

de es ta forma de la recta conocemos un punto $P(x_0,y_0,z_0)$ y su vector director $\vec{U}=(a,b,c)$

\subsection{Ecuación paramétrica de la recta en el espacio}

Conocemos las coordenadas de un punto $\color{red}P(x_0, y_0, z_0)$ que pertenece
a la recta. El vector
$\color{blue}\vec{u}=(a,b,c)$ indica hacia donde apunta la recta a partir del punto. Su ecuación es de la forma:

\begin{empheq}[box=\formulaBox]{equation}
	\begin{cases}
		x &= \color{red}x_0 \color{black}+ \color{blue}a \lambda \\
		y &= \color{red}y_0 \color{black}+ \color{blue}b \lambda \\
		z &= \color{red}z_0 \color{black}+ \color{blue}c \lambda \\
	\end{cases}
\end{empheq}


\subsection{Ecuación continua de la recta en el espacio}
De esta ecuación de la recta conocemos un punto $P(x_0,y_0,z_0)$ y también su vector director
$\vec{U}=(a,b,c)$.

\begin{empheq}[box=\formulaBox]{equation}
	\frac{x-x_0}{a}=\frac{y-y_0}{b}=\frac{z-z_0}{c}
\end{empheq}

% === VECTOR ===
\section{Vector}
Es un conjunto de coordenadas que indican una dirección. Cuando escibimos matemáticas solemos
ponerle encima un "sombrero" para aclarar que nos referimos a un vector. Por ejemplo el
vector $U$ se escribirá como $\vec{U}$.

% === PLANO ===
\section{Plano}

\subsection{Ecuación paramétrica del plano en el espacio}

De un plano conocemos un punto $P(x_0,y_0,z_0)$ y también dos vectores directores que se encuentran
en su interior $\vec{U}=(u_1,u_2,u_3$ y $\vec{V}=(v_1,v_w,v_3)$

\begin{empheq}[box=\formulaBox]{equation}
	\begin{cases}
		x &= x_0 + \lambda u_1 + \mu v_1 \\
		y &= y_0 + \lambda u_2 + \mu v_2 \\
		z &= z_0 + \lambda u_3 + \mu v_3 \\
	\end{cases}
\end{empheq}

\subsection{Ecuación general o implícita del plano en el espacio}
Los coeficientes $A$, $B$ y $C$ indican el vector normal $\vec{n}$ del plano.

\begin{empheq}[box=\formulaBox]{equation}
	Ax+By+Cz+D=0
\end{empheq}

En este ejemplo nos dan el plano $\pi:7x+2y-3z+2=0$ y de él sacamos la información de sus coeficientes
que son $A=7$, $B=2$, $C=-3$ y $D=2$ y por lo tanto el vector normal del plano es $\vec{n}=(7,2,-3)$.



%===================== RECETAS DE GEOMETRÍA =====================
\chapter{Recetas de geometría}

Estas recetas pretenden facilitarte las cosas a la hora de entender y aprobar geometría.

% =============================================
\section{Punto medio de un segmento}\label{punto_medio_segmento}
\begin{itemize}
	\item \textbf{Tengo} dos puntos $\color{red}A(x_1, y_1, z_1)$ y $\color{blue}B(x_2, y_2,z_2)$ 
	\item \textbf{Obtengo} un punto $P_m$ que se encuentra a medio camino entre $\color{red}A$ y
		$\color{blue}B$
\end{itemize}
\begin{enumerate}
\item Aplico la fórmula
\end{enumerate}

\begin{empheq}[box=\formulaBox]{equation}
	P_m=\left(\frac{\color{red}x_1\color{black}+\color{blue}x_2}{2}, 
	\frac{\color{red}y_1\color{black}+\color{blue}y_2}{2}, 
	\frac{\color{red}z_1\color{black}+\color{blue}z_2}{2}\right)
\end{empheq}

\vspace{1cm}

\begin{figure}[H]
	\centering
	\begin{tikzpicture}
		% Segmento de unión 
		\draw [dashed] (-5, 0) -- (5, 0); 
		% Punto Pm
		\draw [color=black, fill=white] (0,0) circle(0.1) node[anchor=north] {$P_m$};
		% Extremo A
		\draw [color=red, fill=red] (-5,0) circle(0.1) node[anchor=east] {$A$};
		% Extremo B
		\draw [color=blue, fill=blue] (5,0) circle(0.1) node[anchor=west] {$B$};
	\end{tikzpicture}
	\caption{Punto medio $P_m$ entre los puntos $\color{red}A$ y $\color{blue}B$}
\end{figure}

% ========================================================
\section{Punto simétrico respecto de un plano}
\begin{itemize}
\item \textbf{Tengo} un punto $\color{red}P$ y un plano $\pi$
\item \textbf{Obtengo} un punto simétrico $\color{blue}P'$ al otro lado del plano. Algo así como un efecto espejo
\end{itemize}
	
\begin{enumerate}
	\item Calculo la recta normal $r$ al plano $\pi$ que pasa por el punto $\color{red}P$
\item Hallo el punto de intersección $P_m$ entre la normal y el plano $\pi$
\item Obtengo el punto simétrico $\color{blue}P'$ mediante la fórmula del punto medio (ver
	\ref{punto_medio_segmento}). Como conozco las coordenadas de un
		punto extremo $\color{red}P$ y también las del punto medio $P_m$ ya solo me queda por obtener el otro
		extremo $\color{blue}P'$.
\end{enumerate}

\begin{figure}[H]
	\centering
	\begin{tikzpicture}
		% Plano
		\draw (0,0) -- (1,2) -- (10,2) -- (11,0) -- (0,0) node[anchor=east] {$\pi$};
		% Punto Pm
		\draw [color=black, fill=white] (5.5,1) circle(0.1) node[anchor=west] {$P_m$};
		% Segmento de unión, recta r
		\draw [dashed] (5.5, 4) -- (5.5, 1)  (5.5, 0) -- (5.5, -4); 
		% Punto P
		\draw [color=red, fill=red] (5.5,3) circle(0.1) node[anchor=east] {$P$};
		% Punto P'
		\draw [color=blue, fill=blue] (5.5,-3) circle(0.1) node[anchor=east] {$P'$};
	\end{tikzpicture}
	\caption{Punto $P'$ simétrico de $P$ respecto del plano $\pi$}
\end{figure}

\paragraph{Ejemplos} (2.4.1) Modelo 2003 Opción A; (2.5.6) Septiembre 2006 Opción A

% ========================================================
\section{Posición relativa de dos rectas}
\begin{itemize}
\item \textbf{Tengo} dos rectas $r$ y $s$ con sus dos vectores directores $\vec{u}$ y $\vec{v}$
respectivamente
\item \textbf{Obtengo} una descripción de como está posicionada una respecto a la otra
\end{itemize}

$$
\begin{cases}
	\vec{u} \propto \vec{v} &
		\begin{cases}
			\text{Tienen al menos un punto en común} \rightarrow \text{\textbf{Coincidentes}} \\
			\text{NO tienen punto en común} \rightarrow \text{\textbf{Paralelas}}
		\end{cases} \vspace{1cm}\\
	\vec{u} \not\propto \vec{v} & 
		\begin{cases}
			\text{Tienen un punto en común} \rightarrow \text{\textbf{Se cortan}} \\
			\text{NO tienen punto en común} \rightarrow \text{\textbf{Se cruzan}}
		\end{cases} \\
\end{cases}
$$

\begin{enumerate}
	\item Si sus vectores directores apuntan en la misma dirección $\vec{u} \propto \vec{v}$
		(proporcionales) averiguar si tienen algún punto en común sustituyendo un punto de la recta
		$r$ en la recta $s$
		\begin{enumerate}
			\item Si el punto pertenece a la recta son coincidentes \textbf{coincidentes}
			\item Si no, son rectas \textbf{paralelas}
		\end{enumerate}
	\item Si apuntan en direcciones diferentes $\vec{u} \not\propto \vec{v}$, 
		averiguar si tienen algún punto en común resolviendo el sistema
		\begin{enumerate}
		\item Si el sistema es compatible determinado tienen un punto común y son
			\textbf{coincidentes}
		\item Si el sistema es incompatible no tienen punto común y son \textbf{paralelas}
		\end{enumerate}
\end{enumerate}

\paragraph{Ejemplos} Problema 2.21.3

% ==========================================================
\section{Distancia entre un punto y una recta}
% Junio 2005 op B (p.212)
\begin{itemize}
	\item \textbf{Tengo} un punto $P$ y una recta $r$ con su vector 
		director $\color{blue}\vec{u}$
	\item \textbf{Obtengo} un número real que indica la distancia
\end{itemize}

\begin{enumerate}
	\item Calculo el vector punto a punto $\color{red}\vec{PP}$, que va desde el punto $P$ a un punto
		cualquiera de la recta $r$
	\item Uso la fórmula
\end{enumerate}

\begin{empheq}[box=\formulaBox]{equation}
	d=\frac{|\color{red}\vec{PP} \times \color{blue}\vec{u}|}{|\color{blue}\vec{u}|}
\end{empheq}

\paragraph{Ejemplo} Problema 2.22.2
% ====================================================
\section{Distancia entre un punto y un plano}
\begin{itemize}
	\item \textbf{Tengo} un punto $\color{blue}P$ y un plano $\pi:\color{red}Ax+By+Cz+D\color{black}=0$
	\item \textbf{Obtengo} un número real que indica la distancia
\end{itemize}

\begin{enumerate}
\item El numerador de la fórmula contiene la ecuación del plano en valor absoluto.
	El denominador contiene el módulo de la normal al plano

	\begin{empheq}[box=\formulaBox]{equation}
		d=\frac{|
		\color{red}A\color{blue}x\color{black}+
		\color{red}B\color{blue}y\color{black}+
		\color{red}C\color{blue}z\color{black}+D|}
		{\sqrt{\color{red}A^2+B^2+C^2}}
	\end{empheq}

\item Sustituyo las variables $x$, $y$ y $z$ por las coordenadas del punto $\color{blue}P$
\end{enumerate}

\paragraph{Ejemplo} Problema 2.20.4
% =====================================================
\section{Distancia entre dos rectas}
\begin{itemize}
	\item \textbf{Tengo} dos rectas $r$ y $s$
	\item \textbf{Obtengo} un número real que indica la distancia
\end{itemize}
\begin{enumerate}
	\item Obtener un vector de cada recta $\vec{u}$, $\vec{v}$
	\item Obtener un punto de cada recta que llamaremos $P_r$ y $P_s$ y construimos el vector punto
		a punto $\vec{PP}$
	\item Aplicamos la fórmula. En el numerador se hace el producto mixto, en el denominador el
		módulo del producto vetorial
\end{enumerate}

\begin{empheq}[box=\formulaBox]{equation}
	d_{r, s}=\frac{|[\vec{PP}, \vec{u}, \vec{v}]|}{|\vec{u}\times\vec{v}|}
\end{empheq}

\paragraph{Nota} Si dos rectas se cortan o son coincidentes, su distancia es nula (0)

\paragraph{Ejemplo} Problema 2.21.1
% ================================================
\section{Área del paralelogramo encerrado por dos vectores}
\begin{itemize}
\item \textbf{Tengo} dos vectores $\vec{u}$ y $\vec{v}$
\item \textbf{Obtengo} un número real que indica el área en unidades al cuadrado
\end{itemize}

\begin{enumerate}
\item Aplico la fórmula del área, que consiste en hacer el producto vectorial y luego su módulo
	\begin{empheq}[box=\formulaBox]{equation}
		A=|\vec{u}\times\vec{v}|
	\end{empheq}

\end{enumerate}
\paragraph{Nota} Si se trata de un triángulo, divido su área a la mitad

% ==================================================
\section{Construir recta a partir de dos planos}
\begin{itemize}
	\item \textbf{Tengo} dos planos $\pi_1$ y $\pi_2$. Para construir una recta debemos obtener dos cosas: un vector
	\item \textbf{Obtengo} un vector director $\vec{u}$ y un punto $P$ con el que contruir una recta $r$
\end{itemize}

\begin{enumerate}
	\item Hago el producto vectorial de las normales de cada plano
	\item El resultado será el vector director $\vec{u}$ de la recta que busco
	\item Ahora debemos obtener el punto, para ello doy un valor a una de las incógnitas, por
		ejemplo $x=0$ y resolvemos el sistema de ecuaciones para obtener las otras dos incógnitas.
		Los tres resultados $(x,y,z)$ son las coordenadas del punto $P$
	\item Escribirmos la recta en paramétricas con el punto $P$ y su vector director $\vec{u}$
\end{enumerate}


% =========================================================
\section{Construir plano que contiene recta y punto}
\begin{itemize}
	\item \textbf{Tengo} un punto $\color{red}P(x,y,z)$ y una recta $r$
	\item \textbf{Obtengo} un plano $\pi$ en forma general, también conocida como ecuación implícita
\end{itemize}

\begin{enumerate}
	\item Saco el vector director $\color{blue}\vec{U}=(u_x,u_y,u_z)$ de la recta $r$
	\item Obtengo un punto de la recta al que llamaré $P_r$
	\item Contruyo el vector punto a punto $\color{orange}\vec{PP}=(PP_x,PP_y,PP_z)$ que va desde el punto $P$ a $P_r$
	\item Aplico la fórmula que nos da el plano
\end{enumerate}

\begin{empheq}[box=\formulaBox]{equation}
\begin{vmatrix}
	x-\color{red}P_x & y-\color{red}P_y & z-\color{red}P_z \\
	\color{blue}U_x & \color{blue}U_y & \color{blue}U_z \\
	\color{orange}PP_x & \color{orange}PP_y & \color{orange}PP_z
\end{vmatrix}
=0
\end{empheq}

\begin{figure}[H]
	\centering
	\begin{tikzpicture}
		% Plano
		\draw (0,0) -- (1,3) -- (10,3) -- (11,0) -- (0,0) node[anchor=east] {$\pi$};
		% Recta contenida
		\draw [dotted, line width=1pt, color=black] (4, 4) -- (10, -2) node[anchor=west] {$r$};
		\draw [->, line width=2pt, color=blue] (5.2, 2.8) -- (7.6, 0.4) node[anchor=west] {$\vec{U}$};
		% Punto P
		\draw [color=red, fill=red] (3,1) circle(0.1) node[anchor=east] {$\color{red}P$};
		% Vector punto a punto
		\draw [->, line width=2pt, color=orange] (5.5, 2.5) -- (3,1) node[below, midway] {$\vec{PP}$};
		% Punto Pr
		\draw [color=black, fill=black] (5.5,2.5) circle(0.1) node[anchor=west] {$P_r$};
	\end{tikzpicture}
	\caption{Plano a partir de una recta contenido en él y un punto $\color{red}P$}
\end{figure}
% ====================================================
\section{Construir plano a partir de vector perpendicular y punto}

\begin{itemize}
	\item \textbf{Tengo} un vector perpendicular $\color{red}\vec{n}=(A,B,C)$. Este vector será la normal del plano.
		Tengo también un punto $\color{blue}P$
	\item \textbf{Obtengo} un plano $\pi$
\end{itemize}

\begin{enumerate}
	\item Todos los planos son de la forma
		$Ax+By+Cz+D=0$. Usaré los coeficientes 
		$\color{red}A,B,C$ de la normal $\color{blue}\vec{n}$ en la ecuación del plano.
	\item Para averiguar la incógnita $D$, sustituyo $x$,$y$ y $z$ del plano por las coordenadas
	del punto $\color{blue}P$.
\end{enumerate}

\begin{figure}[H]
	\centering
	\begin{tikzpicture}
		% Plano
		\draw (0,0) -- (1,2) -- (10,2) -- (11,0) -- (0,0) node[anchor=east] {$\pi$};
		% Vector normal
		\draw [->, line width=2pt, color=red] (5.5, 1) -- (5.5, 3) node[anchor=west] {$\vec{n}$};
		% Punto
		\draw [color=blue, fill=blue] (3,1) circle(0.1) node[anchor=east] {$P$};
	\end{tikzpicture}
	\caption{Plano a partir de un vector perpendicular $\color{red}\vec{n}$ y un punto
	$\color{blue}P$}
\end{figure}

\begin{center}
\begin{tabular}{c}
	\qrcode{https://samuelgomez.es/r/2020/1}\\
	Ejemplo \thesection.1
\end{tabular}
\end{center}


% ======================================================
\nocite{santillana2016_2bach}
\nocite{cms1}
\printbibliography



\end{document}

