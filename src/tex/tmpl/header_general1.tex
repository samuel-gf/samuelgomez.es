%\documentclass[12pt,monochrome]{book}
\documentclass[12pt]{book}
\usepackage{mathtools}
\usepackage[utf8]{inputenc}                         % UNICODE
\usepackage[T1]{fontenc}                            % Permite copiar caractéres unicode desde el PDF al portapapeles
\usepackage{amsmath,amsthm}                         % Matemáticas American Mathematical Society
\usepackage{amssymb}                                % Math symbols
\usepackage{graphicx}                               % Insertar gráficos
\usepackage[ddmmyyyy]{datetime}                     %
\usepackage[spanish, es-nodecimaldot]{babel}        % Corte de las palabras en español
\usepackage{tikz}
\usetikzlibrary{babel}								% Imprescindible para que tikz funcione con babel
\usetikzlibrary{mindmap,trees}
\usepackage{enumitem}
\setenumerate[0]{label=\alph*)}
\usepackage[hidelinks]{hyperref}                    % Enlaces sin apariencia fea
\usepackage{gensymb}                                % Símbolos de grado
\usepackage[pscoord]{eso-pic}                       % Overlay text. The zero point of the coordinate system is the lower left corner of the page (the default).
\usepackage{xcolor}                                 % Texto con diferentes colores
%\selectcolormodel{gray}
\usepackage{siunitx} \sisetup{ output-decimal-marker={,}, quotient-mode=fraction}   % Sistema internacional para unidades con separador decimal
\usepackage{relsize}								% Integrales grandes
\usepackage{qrcode}
\usepackage{sectsty}								% Estilo para las secciones
\usepackage{empheq}									% Ecuaciones en cajas
\usepackage[pagestyles,innermarks]{titlesec}		% Configuración de títulos
\usepackage{float}									% Posición figuras con [H]
%\usepackage{chngcntr}								% Resetea contador de ecuaciones
%\usepackage{xpatch}
%\usepackage{everyshi}								% Contador se incrementa al cambiar de página
\usepackage{imakeidx}
\makeindex

% == Data of the document ==
\date{\today}
\author{Samuel Gómez Fernández}

% == Teoremas ==
\newtheoremstyle{named}{\topsep}{\topsep}{\itshape}{}{\bfseries}{.}{.5em}{\thmname{#1}
	\roman{chapter}\thmnumber{#2} \thmnote{#3}}
\theoremstyle{named}
\theoremstyle{plain}
	\newtheorem{propo}{Proposición}[section]
	\newtheorem{teor}[propo]{Teorema}
	\newtheorem{coro}[propo]{Corolario}
	\newtheorem{lema}[propo]{Lema}
\theoremstyle{definition}
	\newtheorem{defi}{Definición}[chapter]
	\newtheorem{ejem}{Ejemplo}
	\newtheorem{ejer}{Ejercicio}
\theoremstyle{remark}
	\newtheorem*{nota}{Nota}
	\newtheorem*{notac}{Notación}

% Cabecera ===========================================
\newpagestyle{DocumentoGeneral}{
	\headrule
	\sethead{
		\ifnum\value{chapter}>0
			Capítulo \roman{chapter} 
		\else
		\fi
		\hfill \MakeUppercase{\chaptertitle}
		}{}{}
	\footrule\setfoot{}{\usepage{}}{}
	}


% Redefino nombre y número de figura ==================
\renewcommand\thefigure{\roman{chapter}.\arabic{figure}}
%\counterwithin{equation}{section}

% Bibliografía ========================================
\usepackage[backend=biber,style=alphabetic,sorting=nty,citestyle=alphabetic]{biblatex}
\bibliography{bib/matematicas.bib}{}
\setlength\bibitemsep{1.5\itemsep}
\DefaultInheritance{all=true}
%\setlength\bibitemsep{\baselineskip}
\usepackage{authblk}	% Multiple authors

% Entorno problema ========================================
\newcounter{problema}[section]
\newenvironment{problema}[2]
{\refstepcounter{problema} \noindent\textbf{Problema \thesection.\theproblema} \hspace{0.5cm} #2 
	\begin{flushright}\textit{#1}\end{flushright} 
	\paragraph{Solución}}
{\vspace{1cm}}

% Fórmulas enfatizadas ======================================
\definecolor{formulaFondo}{HTML}{e0e0ff}
\newcommand*\formulaBox[1]{\colorbox{formulaFondo}{\hspace{1em}#1\hspace{1em}}}

% Entorno lista de propiedades =================================
\newlist{propiedades}{enumerate}{2}
\setlist[propiedades]{label=Propiedad \arabic*.,itemindent=*}

% Formato de capítulos =========================================
\definecolor{tituloColor}{HTML}{006aff}
%\newcommand{\chapterbreak}{\clearpage}

\newcommand{\chapterFormato}[1]{
	\huge\textcolor{tituloColor}{#1}\\
	\vspace{1cm}
	\color{tituloColor}\titlerule[1pt]
}
\titleformat{\chapter}[display]{\filright\normalfont\bfseries}
	{\textcolor{tituloColor}{	% El número de la sección
		Capítulo \roman{chapter}}\;\;
		\color{tituloColor}\titlerule[1pt]
		%\setcounter{pageInChapter}{1}
	}	
	{0em}	% Distancia entre número y texto de título de sección
	{\chapterFormato}	% El texto de la sección


% Formato de secciones =========================================
\newcommand{\sectionFormato}[1]{
	\large\textcolor{tituloColor}{#1}
}
\titleformat{\section}[hang]{\normalfont\bfseries}
	{\textcolor{tituloColor}{	% El número de la sección
		\Large\thesection.}
	}	
	{0em}	% Distancia entre número y texto de título de sección
	{\sectionFormato}	% El texto de la sección
\renewcommand{\thesection}{\arabic{section}}

% Formato de subsecciones =========================================
\newcommand{\subsectionFormato}[1]{
	\large\textcolor{black}{#1}
}
\titleformat{\subsection}[hang]{\normalfont\bfseries}
	{\textcolor{black}{	% El número de la sección
		\large\thesubsection.}
	}	
	{0em}	% Distancia entre número y texto de título de sección
	{\subsectionFormato}	% El texto de la sección
% QR CODE ===========================================
\qrset{height=2cm,padding}

